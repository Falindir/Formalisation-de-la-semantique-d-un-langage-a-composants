  
  Le Génie Logiciel (GL) est une discipline de l'informatique visant à proposer des modèles, logiciels, méthodes et outils pour la production de logiciels, reposant sur un développement standardisé, automatisé et organisé dans le but de réduire les coûts de production, permettant aux créateurs de logiciels de maîtriser leurs produits sur toutes leurs facettes~: fonctionnalités, fiabilité, qualité, maintenabilité et réutilisé \cite{dony1989langages}. On trouvera une définition du GL dans \cite{Boehm:1976:SE:1311958.1312684}; une autre définition de S.Krakowiak : 
 
   \begin{quote}
      \emph{ ... ensemble des méthodes, techniques, et outils nécessaires à la
production de logiciel de qualité industrielle, pour l'ensemble du cycle
de vie d'un produit... Il s'agit en fait de passer à terme d'une technique
basée sur la réparation, ... à une technique fondée sur la conception sûre
et la garantie à priori de la qualité.} 
  \end{quote}

  mettant en avant les problématiques de modularité (décomposer un programme en modules -- fonctions et/ou méthodes et/ou sous-programmes -- afin de réaliser un développement indépendant et de répartir la complexité du programme) et de réutilisa\-bilité (réutiliser tout ou partie d'un logiciel -- spécification et/ou conception et/ou programme -- pour en faire un autre). Le développement par objets est une partie importante du GL, permettant de répondre à ces problématiques, mais n'est pas le seul paradigme possible.  En effet le développement par composants permet lui aussi de répondre à ces problématiques.\\\par
  
  Dans ce contexte global nous nous intéressons plus spécifiquement aux architectures logicielles à base de composants.
  Le concept de composant a été introduit historiquement par \cite{Il68}, donnant depuis de multiples recherches, pouvant être regroupées en trois grandes familles : les frameworks, les approches génératives et les langages à composants (LOC), s'unifiant autour d'une idée centrale, qui voit le composant comme une entité boîte noire, communiquant avec le monde extérieur au moyen d'interfaces (requises ou fournies). \\\par
  
  Dans ce stage, nous allons nous intéresser à la dernière famille. L'approche des LOC propose la création et la manipulation de composants au sein d'un seul et même langage de programmation. Les langages de cette famille se divisent en deux catégories. Une première dite <<mixte>>, évoluant dans un monde où la notion d'objet et de composant cohabitent, c'est-à-dire où les composants s'échangent des données au travers d'objets. La deuxième dite <<pure>> où seule la notion de composant existe (cf. annexe \ref{ann:compoexemple}). Les composants s'échangeant des données par connexion à d'autres composants. \\\par
  
  Afin d’enlever toute ambiguïté dans la spécification d'un langage de programmation définie dans le langage naturel, on peut définir formellement la sémantique de ce langage. Cela permet de caractériser mathématiquement les calculs décrits par un programme et les résultats qu'il produit. Les programmes deviennent des objets mathématiques qui doivent s'évaluer selon un certain nombre de règles définies formellement par la sémantique du langage.\\\par
  
  La problématique du stage se place dans ce contexte spécifique des langages pur composants. Nous allons nous placer dans le cadre particulier du langage Compo, développé au sein de l'équipe MAREL du LIRMM. L'objectif du stage est de définir la séman\-ti\-que formelle de ce langage. Cependant, avant de mettre en place cette sémantique, nous allons compléter la conception de Compo en étudiant les différents modes de passage de paramètres dans un monde <<tout composant>>. \\\par
  
  Dans le chapitre suivant, nous aborderons un état de l'art autour des trois grandes familles à composants, les différents modes de passage de paramètres dans les langages et la formalisation de la sémantique d’un langage. Le chapitre \ref{chap:problematique} présentera les différentes problématiques liées au sujet ainsi que les pistes de recherche envisagées dans le cadre de l'évolution de la conception de Compo. \\\par
  
  
  
