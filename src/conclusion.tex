
 Nous avons vu au cours de cette étude les différentes familles à composants qui se regroupent autour d'une idée commune, voyant un composant comme étant une entité boîte noire, communiquant avec le monde extérieur au moyen d'interfaces (requises ou fournies). Cette idée de composant apporte au Génie Logiciel une nouvelle approche pour répondre aux problématiques de modularité et de réutilisabilité dans la production de logiciels.
 
 Nous avons aussi vu que la formalisation de la sémantique d'un langage permet de définir le comportement d'un programme de manière mathématique. Les motivations sont nombreuses, allant de la visualisation de tout comportement possible à l'exécution, en passant par la lever de toute ambiguïté pour un utilisateur du langage, jusqu'à des propriétés relatives au typage. Cependant cette formalisation de la sémantique d'un langage à composants est à l'heure actuel et à notre connaissance inexistante. 
 
  Nous avons présenté le langage Compo qui se place dans le contexte d'un langage pur composants qui a pour but d'unifier les concepts des différentes familles à composants, autour d'un seul langage de programmation qui met en avant la notion de <<tout composant>>. Cette notion apporte de nouvelles problématiques, notamment le passage de paramètres. En effet, nous avons vu que les autres familles à composants et plus particulièrement les LOCs, permettent aux composants de s'échanger des données entre eux qui sont des objets, qui dans le langage Compo permet aux composants de s'échanger des données par connexion à d'autres composants.

 Cela pose donc la question de savoir la nature de cette connexion et les contrôles qu'ont les composants sur elle. Les premières idées qui sont présentes pour l'instant mettent en avant un passage par requis et un passage par fournis. Cependant nous avons vu que ces passages n'ont pas encore atteint leur plein potentiel, notamment des potentialités mis en avant par le passage par requis (passage en lecture seule et passage révocable). Les passages de paramètres classiques sont aussi une source d'inspiration, avec des premières similarités avec le passage par référence et la correspondance avec les autres passages dans le monde des composants est aussi une bonne piste pour avancer dans cette problématique.
 
   L'objectif durant le stage sera donc une approche originale autour de la formalisation de la sémantique d'un langage à composants, mécanisé avec l'outil Coq, tout en proposant une implémentation d'un ou plusieurs modes de passage de paramètres dans un monde <<tout composant>>, par l'évolution d'un passage existant et/ou la création d'un ou plusieurs nouveaux modes de passages de paramètres.
   

   
   
   
   
