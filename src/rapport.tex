

%% La classe stageM2R s'appuie sur la classe memoir, plus d'information sur le paquet: http://www.ctan.org/pkg/memoir
%% option possible de la classe stageM2R
% utf8  -> encodage du texte UTF8 (défaut: Latin1)
% final -> mode rapport de stage final (défaut: mode étude bibliographique)
% private -> indique une soutenance privée (défaut: soutenance publique)
\documentclass[utf8]{stageM2R} %-> etude bibliographique
%\documentclass[utf8,final]{stageM2R} %-> rapport final

%%%%%%%%%%%%%%%%%%%%%%%%%%%%
%%% Déclaration du stage %%%
%%%%%%%%%%%%%%%%%%%%%%%%%%%%

\usepackage{graphicx}
\usepackage{color}
\usepackage{listings}
\usepackage{bussproofs}
\usepackage{textcomp}

%% auteur
\author{Jimmy Lopez}
%% encadrants
\supervisors{David Delahaye\\Christophe Dony\\Chouki Tibermacine}
%% lieu du stage (Optionnel)
\location{LIRMM UM5506 - CNRS, Université de Montpellier}
%% titre du stage
\title{Formalisation de la sémantique d’un langage à composants} 
%% parcours du master
\track{AIGLE}  
%% date de soutenance (Optionnel)
\date{\today} 
%% version du rapport (Optionnel)
\version{1}
%% Résumé en francais
%%\abstractfr{
%%Ce stage de master.
%%}
%% Résumé en anglais
%%\abstracteng{
 %% This master thesis.
%%}

\setsecnumdepth{subsubsection}
\setcounter{tocdepth}{2}

\hypersetup{
  hidelinks
}

\begin{document}   
%\selectlanguage{english} %% --> turn the document into english mode (Default is french)
\selectlanguage{french} 
\frontmatter  %% -> pas de numérotation numérique
\maketitle    %% -> création de la page de garde et des résumés
\cleardoublepage   
\tableofcontents %% -> table des matières
%\clearpage 
%\listoffigures %% -> figures
%\clearpage
%\lstlistoflistings %% -> listing
\mainmatter  %% -> numérotation numérique

\lstset{ %
language=java,                % choose the language of the code
basicstyle=\footnotesize,       % the size of the fonts that are used for the code
numbers=left,                   % where to put the line-numbers
numberstyle=\footnotesize,      % the size of the fonts that are used for the line-numbers
stepnumber=1,                   % the step between two line-numbers. If it is 1 each line will be numbered
numbersep=5pt,                  % how far the line-numbers are from the code
backgroundcolor=\color{white},  % choose the background color. You must add \usepackage{color}
showspaces=false,               % show spaces adding particular underscores
showstringspaces=false,         % underline spaces within strings
showtabs=false,                 % show tabs within strings adding particular underscores
frame=single,           % adds a frame around the code
tabsize=2,          % sets default tabsize to 2 spaces
captionpos=b,           % sets the caption-position to bottom
breaklines=true,        % sets automatic line breaking
breakatwhitespace=false,    % sets if automatic breaks should only happen at whitespace
escapeinside={\%*}{*)}          % if you want to add a comment within your code
}

%%%%%%%%%%%%%%%%%%%%%%%%%%%%%%
%%%%    DEBUT DU RAPPORT  %%%%
%%%%%%%%%%%%%%%%%%%%%%%%%%%%%%

\chapter{Introduction}

    
  Le Génie Logiciel (GL) est une discipline de l'informatique visant à proposer des modèles, logiciels, méthodes et outils pour la production de logiciels, reposant sur un développement standardisé, automatisé et organisé dans le but de réduire les coûts de production, permettant aux créateurs de logiciels de maîtriser leurs produits sur toutes leurs facettes~: fonctionnalités, fiabilité, qualité, maintenabilité et réutilisé \cite{dony1989langages}. On trouvera une définition du GL dans \cite{Boehm:1976:SE:1311958.1312684}; une autre définition de S.Krakowiak : 
 
   \begin{quote}
      \emph{ ... ensemble des méthodes, techniques, et outils nécessaires à la
production de logiciel de qualité industrielle, pour l'ensemble du cycle
de vie d'un produit... Il s'agit en fait de passer à terme d'une technique
basée sur la réparation, ... à une technique fondée sur la conception sûre
et la garantie à priori de la qualité.} 
  \end{quote}

  mettant en avant les problématiques de modularité (décomposer un programme en modules -- fonctions et/ou méthodes et/ou sous-programmes -- afin de réaliser un développement indépendant et de répartir la complexité du programme) et de réutilisa\-bilité (réutiliser tout ou partie d'un logiciel -- spécification et/ou conception et/ou programme -- pour en faire un autre). Le développement par objets est une partie importante du GL, permettant de répondre à ces problématiques, mais n'est pas le seul paradigme possible.  En effet le développement par composants permet lui aussi de répondre à ces problématiques.\\\par
  
  Dans ce contexte global nous nous intéressons plus spécifiquement aux architectures logicielles à base de composants.
  Le concept de composant a été introduit historiquement par \cite{Il68}, donnant depuis de multiples recherches, pouvant être regroupées en trois grandes familles : les frameworks, les approches génératives et les langages à composants (LOC), s'unifiant autour d'une idée centrale, qui voit le composant comme une entité boîte noire, communiquant avec le monde extérieur au moyen d'interfaces (requises ou fournies). \\\par
  
  Dans ce stage, nous allons nous intéresser à la dernière famille. L'approche des LOC propose la création et la manipulation de composants au sein d'un seul et même langage de programmation. Les langages de cette famille se divisent en deux catégories. Une première dite <<mixte>>, évoluant dans un monde où la notion d'objet et de composant cohabitent, c'est-à-dire où les composants s'échangent des données au travers d'objets. La deuxième dite <<pure>> où seule la notion de composant existe (cf. annexe \ref{ann:compoexemple}). Les composants s'échangeant des données par connexion à d'autres composants. \\\par
  
  Afin d’enlever toute ambiguïté dans la spécification d'un langage de programmation définie dans le langage naturel, on peut définir formellement la sémantique de ce langage. Cela permet de caractériser mathématiquement les calculs décrits par un programme et les résultats qu'il produit. Les programmes deviennent des objets mathématiques qui doivent s'évaluer selon un certain nombre de règles définies formellement par la sémantique du langage.\\\par
  
  La problématique du stage se place dans ce contexte spécifique des langages pur composants. Nous allons nous placer dans le cadre particulier du langage Compo, développé au sein de l'équipe MAREL du LIRMM. L'objectif du stage est de définir la séman\-ti\-que formelle de ce langage. Cependant, avant de mettre en place cette sémantique, nous allons compléter la conception de Compo en étudiant les différents modes de passage de paramètres dans un monde <<tout composant>>. \\\par
  
  Dans le chapitre suivant, nous aborderons un état de l'art autour des trois grandes familles à composants, les différents modes de passage de paramètres dans les langages et la formalisation de la sémantique d’un langage. Le chapitre \ref{chap:problematique} présentera les différentes problématiques liées au sujet ainsi que les pistes de recherche envisagées dans le cadre de l'évolution de la conception de Compo. \\\par
  
  
  

  
\chapter{État de l'art}

La première section positionne le concept des langages de programmation par composants parmi les autres approches relevant de ce concept en GL. La deuxième section nous replace dans le contexte des langages de programmation et de leurs différentes techniques permettant le passage d'arguments. Enfin la dernière section nous place dans le contexte de la définition de la sémantique des langages de programmation, avec la mise en place d'un mécanisme de preuve. 

  \section{Le développement logiciel à base de composants}
  
    \subsection{Présentation de l'approche composant}
    
Le terme de «~composant~», définit dans l'approche de l'ingénierie du logiciel basée sur les composants (\emph{Component-Based Software Engineering -- CBSE}) étant très générique, en donner une définition exacte et précise paraît difficile car dépend fortement du contexte de son utilisation. Cependant on peut se baser sur des définitions faites dans la littérature :

\begin{quote}
  \emph{ A software component is a unit of composition with contractually specified interfaces and explicit context dependencies only. A software component can be deployed independently and is subject to composition by third parties.} \cite{Szyperski:2002:CSB:515228}
\end{quote}
  
\begin{quote}
  \emph{ A component is a nontrivial, nearly independent, and replaceable part of a system that fulfills a clear function in the context of a well-defined architecture. A component conforms to and provides the physical realization of a set of interfaces.} \cite{kruchten1998modeling}
\end{quote}
  
\begin{quote}
  \emph{ A component is a unit of distributed program structure that encapsulates its implementation behind a strict interface comprised of services provided by the component to other components in the system and services required by the component and implemented elsewhere. The explicit declaration of a component's requirements increases reuse by decoupling components from their operating environment.} \cite{pryce1998component}
\end{quote}
  
Ces différentes définitions permettent de faire ressortir des notions qui se retrouvent dans la plupart des approches à composants : 
    
\begin{description}

  \item[composition] Le mécanisme de composition permet de créer à partir d'un assemblage de << connexion >> de composants, un nouveau composant plus complexe, en encapsulant des composants afin de pouvoir réutiliser directement cette assemblage. Ce nouveau composant est dit composite car composé de composants qui deviennent des sous-composants (composants interne), pouvant être eux aussi des composants composites ou des composants primitifs \footnote{un composant est dit primitif s'il ne contient pas de sous-composants}.
  
  \item[interfaces] C. Szyperski définit une interface d'un composant comme étant un point d'accès au service du composant \cite{szyperski1999components} permettant de décrire comment peuvent être assemblés nos composants ou utilisés dans notre architecture. On parle d'interfaces << requises >>, pour celles qui permettent de décrire les besoins d'un composant et d'interfaces << fournies >>, les interfaces permettant de définir les fonctionnalités que proposent notre composant aux autres composants.
  
  \item[architecture] La notion d'architecture permet de représenter le plan de notre application et permet de décrire comment doit être conçu notre application afin de respecter les spécifications mises en place.
 
  \item[service] Les services permettent de représenter la logique métier d'un composant. Géné\-ralement présent uniquement dans des composants primitifs, certaines approches essaient de définir des services pour des composants composites. 
  
  \item[indépendance] Il faut voir un composant comme un élément indépendant de tous systèmes. Il doit être assez générique pour pouvoir se connecter à d'autres composants au sein d'une nouvelle application en fournissant un ensemble de services, sens être trop spécifique à un système afin de répondre à un besoin précis.
  
\end{description}
    
    
\subsection{Les principes de l'approche à composant}

  De nombreux concepts permettent la notion de réutilisabilité pour le développement (appel de fonction, importation de module, héritage entre classe, paramétrage de framework, assemblage de composants, etc.). Cependant même si ses différentes techniques permettant de mettre en avant cette notion de réutilisabilité, les composants on fait de la réutilisation le fer de lance de leurs approches.\\\par  
    
  On identifie deux nouveaux acteurs de l'approche à composants \cite{fabresse2007decoupage} : le dévelop\-peur de composants et l’architecte d’application (cf figure \ref{fig:reusecomponent}). Le rôle du développeur va consister à réaliser des composants indépendants. Quant à l'architecte, il récupère les composants déjà créés par un développeur et réalise une intégration dans son assemblage de composants existant. Il est bien sûr possible pour une personne d'avoir les deux casquettes, mais il doit rendre son composant indépendant de l'application. \\\par 
      
\begin{figure}[!t]
\centering
\scalebox{.5}{
\includegraphics{images/reuse.png}
}
\caption{Vision simplifiée du processus de développement par composants, extrait de \cite{fabresse2007decoupage}}
\label{fig:reusecomponent}
\end{figure}
  
  Nous avons vue que l'architecture logiciel permettait de définir comment notre application devait être construite. C'est pour cela que nous avons besoin de visibilité sur cette architecture afin de bien appréhender comment notre application est réalisée et de visualiser toute les interactions entre les éléments de l'application, permettant ainsi de vérifier que cette architecture respecte bien les spécifications. Dans le monde objet, la notion d'architecture est réalisé lors de la phase de conception de l'application et même si le développeur peut s'aider d'outil de représentation graphique, comme UML, cette représentation n'est que peu maintenue lors de la phase d'implémentation, ce qui ne permet plus de voir les modifications apportées par celle-ci et donc n'assure plus que cette nouvelle architecture soit conforme aux spécifications. \\\par
  
  Le monde des composants arrive à corriger cela, en rendant explicite cette visibilité sur l'architecture, soit à la manière des ADLs (Architecture Description Languages), qui permettent de décrire la structure et de définir le comportement des composants à travers leurs assemblages, par une représentation abstraite, soit directement dans l'implémentation des composants par l'approche des langages à composants, qui définissent l'architecture au sein même des composants. Par conséquent, même si pendant la phase d'implémentation on décide de modifier l'architecture de nos composants, on ne perdra pas la visibilité sur l'architecture de notre application car elle est écrite explicitement. \\\par
  
\subsection{Les grandes approches du développement à base de composants}
      
      \subsubsection{La famille des génératives}
      
      La famille des génératives se situe à un haut niveau d'abstraction afin de modéliser et gérer des systèmes logiciels complexes. Cette représentation se base généralement sur les ADLs, permettant de décrire les architectures logiciels à base de composants en y décrivant leurs comportements, leurs interactions avec les autres composants et leurs configurations ainsi que leurs besoins explicites. La stratégie de cette approche générative consiste donc dans la modélisation d'une architecture de composants << initial >> décrite de manière formelle afin de générer un << squelette >> d'application dans un langage de programmation précis qui respectera les spécifications du système. Cette représentation étant abstraite, elle est indépendante des langages de programmation qui implémente les composants, permettant ainsi de s'adapter à tous types de plateforme. \\\par
      
          \textbf{UML -- un langage de modélisation graphique pour composants}
      
      UML (Unified Modeling Language) est un langage de modélisation graphique standardisé par l'OMG (\emph{Objet management groupe}) \footnote{http://www.omg.org/spec/UML}. Dans sa première version, UML intègre déjà la notion de composant, comme étant une entité indépendante et communiquant avec d'autres composants au travers d'interfaces. La version 2.0 d'UML \cite{specificationuml} a permis d'améliorer cette vision des composants, en s'inspirant aussi des mécanismes et concepts des ADLs, introduisant la notion de port, de connecteur et de composant composite.

Un composant UML est donc une entité autonome, qui communiquent avec les autres composants de l'architecture au travers d'interfaces, regroupées dans des ports qui expriment des rôles. Un composant peut disposer de plusieurs ports, où un port regroupe un ensemble d'interfaces requises ou fournies. Le port permet donc de réaliser une communication entre l'environnement extérieur du composant et son architecture interne. 
      
 Cette connexion entre composants est permise par la notion de connecteur, qui permet de représenter les possibilités de communication entre les composants. Il existe deux types de connecteurs : d’assemblage et de délégation. Un connecteur d'assemblage permet de lier deux interfaces de deux composants. Avec une interface requise pour notre premier composant et une interface fournie pour notre deuxième composant. Cette connexion ne peut se faire que si les deux interfaces ont la même signature. Un connecteur de délégation quant à lui est utilisé afin de réaliser une connexion entre ses interfaces ou ports vers les interfaces ou ports de ses composants internes, permettant ainsi un délégation d'un besoin ou d'un service vers un composant interne.
        
    UML apporte cet aspect visuel à la représentation des architectures composantes qui manquent aux autres approches, aidant ainsi un architecte à mieux visualiser l'architecture qu'il est en train de construire, mais ne permet pas de réaliser l'implémentation et la manipulation de composants sans le coupler à une autre technique. \\\par 
    
        \textbf{Fractal -- un framework orienté composants intégrant un ADL}
        
        Fractal \cite{bruneton2006fractal} est une spécification d'un modèle à composant proposé par le consortium \emph{ObjectWeb} en 2002, proposant ainsi un framework pour la création d'applications déployables sur un serveur et proposant la création et la manipulation de composant, intégrant aussi un ADL pour la description de l'architecture des composants. Il met en avant le concept de composant composite, de composant partagé, un mécanisme d'introspection et la possibilité de configuration dynamique. 
        
        Un composant Fractal est composé de deux partie, la partie \emph{contenu} qui va contenir les fonctionnalités que propose le composant, en étant soit un composant primitif qui dans son \emph{contenu} va implémenter la logique métier, soit un composant composite qui encapsule un assemblage de composants qui réalise les fonctionnalités attendues. La deuxième partie d'un composant est appelé la \emph{membrane}, qui contient les parties non-fonctionnelles du composant, intégrant les différentes interfaces qui communique avec le monde extérieur, ainsi que les interfaces qui permettent le contrôle du composant.
        
        Une interface a deux type de visibilité : externe qui est visible depuis l’extérieur du composant et permettra une liaison avec d'autre composant et interne permettant une liaison avec son contenu. Une interface peut être de différents rôles, serveur qui est une interface fournie proposant les différents services du composants, cliente qui est une interface requise représentant les besoins du composants et contrôle permettant la configuration dynamique du composants avec la possibilité de modifier ces liaisons avec les composants externe, de modifier son assemblage interne (modification des liaisons internes) et la gestion du cycle de vie du composant.
        
        Afin de pouvoir connecter deux composants entre eux, il faut connecter une interface de l'un avec une interface de l'autre, cette connexion est appelée liaison (\emph{binding}) et peut être de trois types : une liaison normale (\emph{normal binding}) entre une interface serveur externe d'un composant et une interface cliente externe d'un autre composant, une liaison d'export (\emph{export binding}) entre une interface cliente d'un composite et un interface serveur d'un composant interne permettant la délégation d'un service fourni par notre composant composite qui sera fourni par son composant interne, et une liaison d'import (\emph{(import binding}) entre une interface serveur interne d'un composant composite et une interface cliente externe d'un composant interne permettant la délégation d'un besoin de notre composant interne au composant composite.
        
        Étant un modèle abstrait, il permet de s'abstraire de tous langages et de plateformes pour la gestion de composants, dont différentes implémentations ont été réalisées (par exemple en Java avec Julia ou en C++ avec Cecilia). En effet l'ADL qu’intègre Fractal est décrit dans une syntaxe XML et spécifié de manière indépendante des langages de programmation. Les différentes implémentation proposent cependant un mécanisme de génération de code à partir de l'ADL pour réaliser l'implémentation des composants. \\\par

    \subsubsection{La famille des frameworks}     
   
    La famille des frameworks à composants utilise les langages de programmation objets pour la création et la manipulation de composants afin de modéliser un système. Cette approche permet au développeur d'utiliser les composants afin de bien séparer les besoins fonctionnels et les besoins non-fonctionnels. Principalement utilisé dans le monde professionnel, les frameworks à composants proposent aux développeurs de créer, déployer et configurer une application.  \\\par
    
    \textbf{Enterprise JavaBeans -- un framework orienté composants pour la création d'applications distribuées} 
    
    L'approche de composant mise en place par Enterprise JavaBeans (EJB), développée au sein de la plateforme Java Enterprise Edition (JEE), propose un modèle à composants serveur permettant la réalisation d'applications distribuées \cite{specificationejb}.
    
    Un composant EJB s’exécute dans un conteneur au sein d'un serveur JEE. Les conteneurs gèrent les instances des EJB, tandis que le serveur fourni aux conteneurs des besoins non-fonctionnels.
    
    Un composants EJB est regroupé en trois catégories : session, entité et message. Une instance d'un composant entité permet de représenter un concept métier, ayant pour but de représenter des données pouvant être stockées dans une base de données. Une instance d'un composant message permet la gestion de messages asynchrones reçu par le serveur depuis le client. Une instance d'un composant session permet de fournir des services définis par des méthodes, pouvant être sans état (stateless) ou avec état (stateful).
    
    Un composant EJB communique ses services par une interface \emph{remote}, spécifiant ainsi les méthodes que fournit une instance d'un EJB au monde extérieur. Il permet sa configuration et la gestion de son cycle de vie par une interface \emph{home} qui est fournie par son conteneur. Cependant on ne peut définir le besoin d'un EJB au moyen d'une interface requise mais les propriétés d'un EJB sont définies dans un descripteur de déploiement permettant au conteneur EJB de savoir comment gérer les composants pendant leur exécution.
    
    L’approche des EJB ne permet pas la réalisation de composant composite. Ainsi une architecture sera dite << à plats >>, chaque composants seront au même niveau, permettant une vision horizontale de l'architecture, mais cela peut s’avérer répétitif de réaliser l'ensemble de toutes les différentes connexions entre composants sans pouvoir utiliser la composition pour réutiliser un assemblage de composants.
         
    \subsubsection{La famille des langages orientés composants}
    
    \label{sec:presentcomponent}
    
    L'approche des langages orientés composants s'articule autour d'un seul langage de programmation pour permettre de définir une architecture de composants et l'implémen\-ta\-tion de ses composants. Ce regroupement de ses deux principes permet une meilleure facilité de développement, le rendant plus naturel car le programmeur ne doit connaître qu'un seul langage de programmation. Elle permet de résoudre les problèmes que pouvaient avoir les ADLs lors de l'implémentation de leur composants, qui pouvaient ne plus respecter les contraintes architecturales en communicant avec d'autres composants de l'application sans être décrites au niveau de l'architecture mais au sein du code source. \\\par 
    
      \textbf{ArchJava -- un langage orienté composants} 
      
      ArchJava \footnote{http://www.archjava.org/} créé en 2001 \cite{aldrich2002archjava}, s'inspire de l'approche composant que proposent les ADLs pour la description d'architecture, tout en permettant l'implémentation de ces composants au sein d'un nouveau langage qui est une extension du langage Java.
      
      Un composant est dans ce monde un <<objet particulier>>, qui s'obtient par instanciation de la classe \texttt{component class}. Cette classe de composant permet de regrouper les informations du composants : ses ports, ses méthodes, ses attributs et permet aussi de décrire son assemblage interne à travers des connexions.
      
  Deux composants ne peuvent communiquer entre eux que s'ils sont explicitement connectés par la primitive \texttt{connect}. Cette primitive introduit pour la première fois dans l'histoire des approches à composant la connexion entre composants par la liaison de leurs ports respectifs. 
  
 Un port définit une liste de méthodes fournies et requises, différenciées respectivement par les primitives \texttt{provides} et \texttt{requires}. Une méthode fournie permet d'introduire une fonctionnalité proposée par le composant au monde extérieur. Les autres composants n'ont accès à cette méthode que par le biais du port. Une méthode requise, quant à elle permet de décrire une fonctionnalité attendue par notre composant. Deux ports ne peuvent se lier entre eux que si les signatures des méthodes requises et fournies qu'ils définissent sont compatibles entre eux.  
      
Une classe de composant peut hériter de toutes les propriétés de la super-classe composant dont elle hérite, que ce soit les méthodes, les ports ou connexions, et permet d'en introduire de nouvelles. On peut redéfinir des méthodes fournies, mais l'ajout de ports ne peut introduire que des méthodes fournies.

  ArchJava offre la compatibilité avec les architectures logiciels Java, au sens où une classe de composant peut hériter d'une classe Java << classique >>. Cependant les contraintes architecturales ne sont plus garanties par ArchJava. Deux composants ne peuvent se passer une référence vers un autre composant, afin de réaliser un transfert de données, toutefois les composants peuvent s'échanger une référence vers un objet, ce qui ne permet plus de visualiser l'architecture logiciel. Un programmeur en ArchJava devra toujours ce demander s'il vaut mieux créer un composant ou un objet pour réaliser une nouvelle fonctionnalité.

         


    
      


  \section{Les différents mécanismes du passage d'arguments en programmation}

    \input{arguments.tex}
  
  \section{La sémantique formelle des langages de programmation}
  
    
  \subsection{Présentation}
  
    La sémantique d'un langage de programmation permet de définir le comportement d'un programme écrit dans ce langage, à la différence de la syntaxe d'un langage de programmation qui permet de décrire la représentation d'un programme dans ce langage. Ces deux notions sont complémentaires mais la frontière entre la sémantique et la syntaxe n'a pas toujours été claire (notamment dans les années 70). Cependant les travaux de C. Strachey et D. Scott ont été fondateurs pour la séparation de ces deux domaines. On peut citer notamment C. Strachey qui disait :
  
  \begin{quote}
      \emph{La sémantique est là pour ce que nous voulons dire et la syntaxe
pour comment nous avons à le dire. } C. Strachey
  \end{quote}
      
  De manière générale, les concepteurs de langage de programmation définissent la sémantique de leur langage de manière informelle, au moyen de blocs de texte, écrits en langage naturel, ce qui laisse la place à l'ambiguïté, lors de l'interprétation d'un comportement décrit dans cette sémantique informelle. Pour éviter toute ambiguïté, les sémantiques formelles reposent sur l'utilisation des mathématiques afin d'exprimer rigoureusement le comportement des programmes.\\\par
        
        Les motivations derrière la formalisation de la sémantique d'un langage sont nombreuses. Cela permet de visualiser tout comportement possible à l'exécution (même les cas d'erreurs), de lever toute ambiguïté pour un utilisateur du langage (car décrit formellement), de démontrer différentes sémantiques pour un même langage, l'équivalence de programmes (syntaxiquement différents), la validation de transformations de programme, des propriétés relatives au typage (comme la correction du typage par rapport à la sémantique ou la préservation du typage lors de l'exécution). Cela permet aussi le développement d'outils certifiés (interprètes, compilateurs, analyseurs statiques, etc.).
      
      \subsection {Les grandes approches de la sémantique formelle}
      
      Dans la littérature on remarque trois grandes approches de la sémantique formelle \cite{Winskel:1993:FSP:151145}~, décrites aussi par Xavier Leroy \footnote{http://cristal.inria.fr/$\sim$xleroy/mpri/prog/cours.pdf}.
      
      \begin{enumerate}
      
        \item La sémantique axiomatique, reposant sur la logique de Floyd-Hoare, permet de décrire le comportement des programmes impératifs annotés par des assertions (formule logique). Elle permet d'assurer la validité des valeurs des variables avant et après l'exécution, si les instructions terminent. 
        
        \item La sémantique dénotationnelle est la première sémantique à avoir été introduit (début 70) par C. Strachey et D. Scott. Elle met en relation une instruction du programme avec sa dénotation. La dénotation est généralement une fonction, au sens mathématique du terme, c'est-à-dire associe des entrées à des sorties (plus généralement un domaine à un autre -- théorie des domaines). Permettant de décrire de façon précise la sémantique du langage mais de façon la plus abstraite possible, elle reste toutefois difficile à manipuler et à implanter. 
        
        \item La sémantique opérationnelle met en relation un programme avec son résultat. Cette relation peut être de deux natures : <<à grands pas>> (dite <<à la Kahn>>) ou <<à petits pas>> (dite <<à la Plotkin>>). Cette sémantique opérationnelle est donnée par un ensemble de règles d'inférence, reposant sur des structures inductives (théorie des types). 
        La sémantique à grands pas met directement en relation un programme avec son résultat, ne montrant pas les étapes intermédiaires de calculs. Elle est sans doute la sémantique la plus proche d'une implantation. En revanche, elle parle plus difficilement des programmes qui ne terminent pas. Cet type de programmes peut être caractérisé plus facilement avec une sémantique à petits pas, qui décrit toutes les étapes intermédiaires de calcul entre le programme et le résultat final. Toutefois, les sémantiques à petits pas restent plus loin d'une implantation et sont moins efficaces en pratique.
      
      \end{enumerate} 
         
    \subsection{Mécanisation des sémantiques formelles}
      
        Pour nous aider dans la formalisation des sémantiques, différents outils d'aide à la preuve existent. Par exemple, Coq \footnote{https://coq.inria.fr/} est un outil développé au sein de l’équipe INRIA $\pi$$r^{2}$ et fondé sur la théorie des types (calcul des constructions inductives). Coq est particulièrement approprié pour formaliser les sémantiques car il propose un très bon support pour les types inductifs. En effet, les types inductifs sont un moyen idiomatique de formaliser en Coq, et permettent en particulier de traduire presque directement les sémantiques opérationnelles exprimées sous la forme de règles d'inférence. Par ailleurs, il existe également des outils en Coq \cite{Rel-Exec3} permettant d'extraire des interprètes à partir de la formalisation de sémantiques exprimées au moyen de types inductifs, ainsi que les preuves de correction correspondantes (c'est-à-dire que les interprètes extraits sont conformes aux sémantiques formalisées). Il existe également d'autres outils d'aide à la preuve qui possèdent un bon support pour l'induction et qui sont aussi appropriés pour la formalisation de sémantiques. On pourra citer en particulier Lego~\footnote{http://www.dcs.ed.ac.uk/home/lego/} ou Isabell~\footnote{https://isabelle.in.tum.de/}.
        
  \subsection{La sémantique formelle des langages à objets et des langages à composants}
  
       Il existe assez peu de travaux sur la formalisation de la sémantique des langages à objets et la référence en la matière sont les travaux de M.~Abadi et L.~Cardelli \cite{Abadi:1996:TO:547964}, qui ont proposé une nouvelle approche dans la compréhension des langages objets, en présentant des calculs d'objets et développant une théorie  autour d'eux, en couvrant leurs sémantiques et leurs règles de typages.
      
      Cependant il n'existe pas, à notre connaissance, de travaux autour de la formalisation de la sémantique des langages à composants, sauf quelques rares exceptions comme~\cite{peschanski2001typeful}.
        
      

        
\chapter{Problématique du stage de recherche} 

  \section{Le langage orienté composant, Compo} 
  
        
    \label{sec:compo} 

    Développé au sein de l'équipe MAREL du LIRMM, le projet Compo\footnote{http://www.lirmm.fr/compo/} est un travail de collaboration et de réflexion autour du développement d'un langages orientés pur composants, promu par son créateur C.Dony avec l'aide de C.Tibermacine et D.Delahaye, ainsi que des travaux de recherche réaliser par P.Spacek \cite{Spacek:2014:CMA:2602458.2602476} et L.Fabresse \cite{fabresse2007decoupage}.
    
    \subsection{Compo -- un langage tout composant}
    
L'idée du projet Compo est d'unifier les concepts des précédentes approches à composants autour d'un langage de programmation. Il récupére des ADLs une description explicite des architectures, ainsi que l'idée de pouvoir définir explicitement le requis d'un composant. Il reprend le concept des LOC de mettre au même niveau une description de l'architecture et l'implémentation des composants. De la même manière que l'approche de UML, il permet une visualisation graphique de l'architecture interne d'un composant. Avec la même philosophie que Smalltalk pour un monde <<tout objet>>, Compo prône un monde <<tout composant>>. Dans sa deuxième version de développement, Compo a intégré la réflexivité au sein de son méta-modèle, réalisée par les travaux de P.Spacek \cite{Spacek:2014:CMA:2602458.2602476}, décrivant les descripteurs, ports et services comme étant des composants. 
  
      \subsubsection{Composants et descripteurs}
    
    Un composants Compo est décrit par son descripteur, dont il est instance. Un descripteur permet de définir la structure et le comportement de chacune de ses instances en décrivant les ports et l'architecture et en définissant les services. 
    
      \subsubsection{Des ports comme seul moyen de communication entre composants}
      
     Un port regroupe un ensemble d'interface (fournies ou requises). Comme l'approche de Fractal, les ports de Compo sont unidirectionnel, c'est-à-dire qu'un port est soit un ensemble d'interfaces fournies, on parle alors de port fourni (\texttt{provides}), soit un ensemble d'interfaces requises, on parle alors de port requis (\texttt{requires}). Contrairement aux ports bidirectionnels, les ports permettent un découplage du requis et du besoin entre composants. En effet les approches comme Archjava ou en UML, les ports sont un ensemble d'interfaces fournies et/ou requises, laissant plus de souplesse dans le développement mais introduit une perte de sécurité au niveau l'architecture. En effet, un composant fournissant un service n'a pas forcement besoin d'avoir accès au service du composant auquel il se connecte. Il possède aussi une visibilité, interne ou externe. Par défaut un port est externe, visible depuis l'environnement extérieur. Un port peut devenir interne, visible uniquement dans l'environnement interne du composant, grâce à l'ajout de la primitive \texttt{internally}.
    
    \subsubsection{Les services}
    
    Un service est représenté par un nom et un ensemble de paramètres. Il représente une fonctionnalité que peut fournir chaque composant qui est instance du descripteur qui le définit. L'invocation du service d'un composant se réalise au travers d'un port requis du composant. Il peut aussi être défini dans des descripteurs de composants composite. Ainsi un composant composite encapsule un assemblage de composants et peut y rajouter des comportements grâce à des services. Ce qui n'est pas possible dans les approches comme Fractal par exemple, où la définition de service ne peut se faire que par des composants primitif.
    
    \subsubsection{Une description d'architecture explicite}
      
      Cette description d'architecture est contenue dans un bloc d'instructions, identifié par la primitive \texttt{architecture}. Elle permet de représenter les connexions entre les composants internes de notre composant avec l’instruction \texttt{connect} et de permettre la délégation de ports, entre un port externe fourni (resp. requis) de notre composant à un port externe fourni (resp. requis) d'un composant interne, avec l'instruction \texttt{delegate}.
      
    \subsubsection{Des composants composites}
      
    Compo intègre lui aussi la notion de composant composite, comme pouvaient le faire certaines des autres approches. Cependant, contrairement à ArchJava qui encapsule les composants internes de la même façon qu'une classe peut le faire avec ses attributs, Compo stocke ses composants internes à l'aide de ports internes requis. Une instance d'un composant interne est liée à ce port requis interne par son port fournis externe.
  
    \subsection{Compo -- une architecture composant d'un compteur}
    
    L'annexe \ref{ann:compoexemple} permet de décrire une architecture de composant qui simule le comportement d'un compteur. Le premier composant est décrit par le descripteur \texttt{Printer}. Il décrit le service \texttt{print}, qui permet d'afficher une chaîne de caractères. Le deuxième composant est décrit par le descripteur \texttt{Counter} qui permet de simuler l'incrémentation d'un compteur au travers de son service \texttt{increment}. Cette même valeur est encapsulée dans notre composant par un composant \texttt{Integer}. Notre composant a aussi besoin d'être connecté à un autre composant lui fournissant le service \texttt{print}.Le dernier composant est décrit par le descripteur \texttt{PrintableCounter} qui décrit notre architecture de composants. Il encapsule nos deux composants et leurs connexions ligne 62. Le service $main$ qu'il fourni, aura comme comportement d'incrémenter le compteur et d'en afficher la valeur.  
    
      
      
      


      
      

    
  \section{La problèmatique du passage des arguments dans les modèles à composants}
    
    \input{probleme_arguments.tex}
    
  \section{Formaliser la sémantique de Compo}
    
      \subsection{Évolution du noyau sémantique existant et formalisation des nouveaux mécanismes}
  
  La première étape de la formalisation de la sémantique du langage Compo, consiste en la définition des règles d'inférence. Partant d'une base de règles déjà définie par D.Delahaye, nous devrons continuer la définition des règles d'inférence pour prendre en compte l'ensemble des mécanismes de Compo, qui se réaliseront sur papier au cours de cette première étape. Cette définition portera cependant sur un noyau de Compo non réflexif, qui sera intégré dans de futurs travaux. \\\par
  
  Nous avons vu précédemment que la réflexion autour de la problématique du passage en argument d'un composant nous suggère différentes pistes. Ainsi la formalisation des différents modes de passage nous apportera un regard neuf dans le choix du mode et de pouvoir comparer les modes entre eux en définissant formellement leur comportement.
    
  \subsection{Mécanisation en Coq et génération d’un interprète certifié}
  
  La deuxième étape sera effectuée avec l'outil Coq, qui mécanisera les règles d'inférence au moyen de types inductifs. Le choix de Coq comme outil pour la mécanisation se justifie par le fait qu'il est particulièrement efficace pour formaliser une sémantique opérationnelle exprimé sous la forme de règles d'inférence. De plus au cours de ma formation j'ai eu location de l'utiliser afin de mécaniser la sémantique d'un mini-langage fonctionnel.
  
  Le but ultime à atteindre serait l'extraction d'un interprète certifié, qui peut se réaliser à l'aide d'un plugin développé par D.Delahaye et un doctorant \cite{Rel-Exec3}.

    
\chapter{Conclusion} 

  
 Nous avons vu au cours de cette étude les différentes familles à composants qui se regroupent autour d'une idée commune, voyant un composant comme étant une entité boîte noire, communiquant avec le monde extérieur au moyen d'interfaces (requises ou fournies). Cette idée de composant apporte au Génie Logiciel une nouvelle approche pour répondre aux problématiques de modularité et de réutilisabilité dans la production de logiciels.
 
 Nous avons aussi vu que la formalisation de la sémantique d'un langage permet de définir le comportement d'un programme de manière mathématique. Les motivations sont nombreuses, allant de la visualisation de tout comportement possible à l'exécution, en passant par la lever de toute ambiguïté pour un utilisateur du langage, jusqu'à des propriétés relatives au typage. Cependant cette formalisation de la sémantique d'un langage à composants est à l'heure actuel et à notre connaissance inexistante. 
 
  Nous avons présenté le langage Compo qui se place dans le contexte d'un langage pur composants qui a pour but d'unifier les concepts des différentes familles à composants, autour d'un seul langage de programmation qui met en avant la notion de <<tout composant>>. Cette notion apporte de nouvelles problématiques, notamment le passage de paramètres. En effet, nous avons vu que les autres familles à composants et plus particulièrement les LOCs, permettent aux composants de s'échanger des données entre eux qui sont des objets, qui dans le langage Compo permet aux composants de s'échanger des données par connexion à d'autres composants.

 Cela pose donc la question de savoir la nature de cette connexion et les contrôles qu'ont les composants sur elle. Les premières idées qui sont présentes pour l'instant mettent en avant un passage par requis et un passage par fournis. Cependant nous avons vu que ces passages n'ont pas encore atteint leur plein potentiel, notamment des potentialités mis en avant par le passage par requis (passage en lecture seule et passage révocable). Les passages de paramètres classiques sont aussi une source d'inspiration, avec des premières similarités avec le passage par référence et la correspondance avec les autres passages dans le monde des composants est aussi une bonne piste pour avancer dans cette problématique.
 
   L'objectif durant le stage sera donc une approche originale autour de la formalisation de la sémantique d'un langage à composants, mécanisé avec l'outil Coq, tout en proposant une implémentation d'un ou plusieurs modes de passage de paramètres dans un monde <<tout composant>>, par l'évolution d'un passage existant et/ou la création d'un ou plusieurs nouveaux modes de passages de paramètres.
   

   
   
   
   

  
\appendix 
  \clearpage
	\phantomsection
	
  \chapter{Une architecture composant d'un compteur en Compo}
    
    \label{ann:compoexemple}
  
    \lstinputlisting[numbers=left,breaklines=true,basicstyle=\footnotesize, caption=Une architecture composant d'un compteur en Compo]{codes/code.compo}
  
\bibliography{reference}
\bibliographystyle{alpha} 
  
\end{document}

    
%%% Local Variables: 
%%% mode: latex
%%% TeX-master: t
%%% End:  
 
